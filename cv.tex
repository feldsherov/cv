\documentclass[12pt]{cutecv}
\usepackage[english]{babel}
\usepackage{hyperref}

\newcommand{\listbullet}{$\; \cdot \;$}

\author{Svyatoslav Feldsherov}
\contacts{svyat@feldsherov.name | +972-53-451-4602 | \href{https://linkedin.com/in/svyat}{linkedin.com/in/svyat} | \href{https://github.com/feldsherov}{github.com/feldsherov}}
\begin{document}

\maketitle

\columnratio{0.3,0.7}
\begin{paracol}{2}
\setlength{\columnsep}{2em}
\setlength{\cvsectionverticalskip}{1mm}
\setlength{\cvinfoverticalskip}{1mm}

\begin{leftcolumn}
\begin{cvsection}{KEY SKILLS}
   SWE \listbullet User-space networking  \\
   Distributed systems \listbullet Concurrency  \\
   Performance \listbullet Algorithms \\
   Backend \listbullet DevOps 
\end{cvsection}

\begin{cvsection}{LANGUAGES}
  C++\listbullet Python \listbullet Go
\end{cvsection}

\begin{cvsection}{TECHNOLOGIES}
  Linux \listbullet Networking \\
  gRPC \listbullet Protobuf \listbullet Docker \\
  Git \\
\end{cvsection}

\begin{cvsection}{FUN FACTS}
  \listbullet \href{https://lore.kernel.org/lkml/CACgs1VAo5AD-6sw2QCNKhRtoOy99XNP24dAWUrdryJKhCxwsMA@mail.gmail.com/}{Fixed (tiny) crash in Linux kernel} \\
  \listbullet \href{https://yatalks.yandex.ru/en/speakers/svyatoslav-feldsherov}{Speaker at YaTalks 2023} \\
\end{cvsection}

\begin{cvsection}{EDUCATION}
  \cvinfobox{MOSCOW STATE UNIVERSITY}{2013-2019 | MS in Math}{GPA: 5.0/5.0}
  \cvinfobox{YANDEX SCHOOL OF DATA ANALYSIS}{2016-2018 | CS Track}
    {Machine Learning and Computer Science school in Moscow.}
  \cvinfobox{TECHSPHERE}{2016-2017 | Data analysis and ML} {Educational project hosted
    by Mail.ru Group and Computer Science faculty
    at Moscow State University.}
\end{cvsection}

\begin{cvsection}{COMPETITIONS}
  \cvinfobox{Russian National Olympiad in Informatics}
  {2011, 2012 | Prizewinner}
  {}
\end{cvsection}

\end{leftcolumn}

\begin{rightcolumn}
\begin{cvsection}{EXPERIENCE}
  \cvinfobox{Google Cloud, Virtual Network Dataplane Telemetry}{ 08.2022-present | Software Engineer}
  {Working on Google Cloud virtual network telemetry. \\
  Main achievements: provided important telemetry for C3 family of VMs, supported Inter-VPC NAT telemetry, 2-10\% fast path packet processing speedup. \\
  Technologies: internal DPDK-like framework, C++ in network data plane plus gRPC and C++ for packet samples processing.}
  
  \cvinfobox{Yandex Search}{ 06.2020-06.2022 | Software Engineer / Team Lead}
  {I was working on a company-wide microservices framework, which serves
  1000+ microservices, 10+ million requests per second.\\
  Main achievements company wide adoption of framework, 
  numerious product launches unblocked by framework features. \\
  Technical achievements: 99.999\% availability, HTTP transport replacement
  with faster implementation, speedup of distributed tracing over our
  framework from minute to 1-3sec per request. \\
  Team was using C++ for data plane, Python for control plane, and tests/infrastructure.}

  \cvinfobox{Yandex Search}{ 11.2018-06.2020 | Software Engineer}
  {Worked on Search Infrastructure team. C++, Python, gRPC, networking.
  Designed and implemented SDCH support for Search Engine Result Page.
  Main part was C++ backend with custom SDCH implementation.\\
  Improved search query preprocessing pipeline. Reimplemented part of business logic with C++, 
  fixed some architectural problems, speedup on 5-10ms on the affected slice of traffic.}
  
  \cvinfobox{Google}{07.2018-10.2018 | Software Engineering Intern}
  {Worked on Shopping Data Quality team. Primary C++, MapReduce, Stubby.
   Designed and implemented an infrastructure for intellectual scheduling of merchants review. This infrastructure helped to improve existing scheduling logic
   and allowed the team to migrate from several legacy services.}
  
   \cvinfobox{Yandex Ads}{03.2017-06.2018 | Junior Backend Developer}
  {Designed and implemented distributed log collection and event
   plotting system using internal technologies, including YTsaurus — Bigtable-like
   storage with MapReduce framework on top of it. \\
   Used coroutine-based web server and batch requests in order
   to handle required RPS on a small number of servers and not to overload
   other services. \\
   This feature enabled the frontend team to log any client-side
   event (usually JS code in a browser) and plot number of any events on
   clients in real time.}

  \cvinfobox{Mail.ru Group, Tarantool}
  {02.2016-12.2016 | Software Engineering Intern}
  {Tarantool is an in-memory NoSQL database implemented in C with some C++.\\
   Implemented a proof of concept version of SQL in Tarantool.
   We combined SQLite parser and execution engine with Tarantool internal structures (B+*-tree).\\
   Fixed some bugs in Tarantool test engine.\\
   Added intrusive heap to Tarantool data structures library.}
\end{cvsection}

\end{rightcolumn}
\end{paracol}

\end{document}
