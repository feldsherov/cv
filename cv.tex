\documentclass[12pt]{cutecv}
\usepackage[english]{babel}
\usepackage{hyperref}

\newcommand{\listbullet}{$\; \cdot \;$}

\author{Svyatoslav Feldsherov}
\contacts{svyat@feldsherov.name | +972-53-451-4602 }
\begin{document}

\maketitle

\columnratio{0.3,0.7}
\begin{paracol}{2}
\setlength{\columnsep}{2em}
\setlength{\cvsectionverticalskip}{1mm}
\setlength{\cvinfoverticalskip}{1mm}

\begin{leftcolumn}
\begin{cvsection}{KEY SKILLS}
  SWE \listbullet User space Networking  \\
   Distributed systems \listbullet Concurency  \\
   Performance \listbullet Algorithms \\
   Backend \listbullet DevOps 
\end{cvsection}

\begin{cvsection}{LANGUAGES}
  C++\listbullet Python \listbullet Go
\end{cvsection}

\begin{cvsection}{TECHNOLOGIES}
  Linux \listbullet TCP IP  \listbullet  HTTP \listbullet  HTTP2 \\
  grpc \listbullet protobuf \listbullet Docker \\
  git \\
\end{cvsection}

\begin{cvsection}{EDUCATION}
  \cvinfobox{MOCSOW STATE UNIVERSITY}{2013-2019 | MS in Math}{GPA: 5.0/5.0}
  \cvinfobox{YANDEX SCHOOL OF DATA ANALYSIS}{2016-2018 | CS Track}
    {Yandex School of Data Analysis is a Machine Learning
     and Computer Science school in Moscow.}
  \cvinfobox{TECHSPHERE}{2016-2017 | Data analysis and ML} {Techsphere is an educational project hosted
    by Mail.ru Group and Computer Science faculty
    at Moscow State University.}
\end{cvsection}


\begin{cvsection}{COURSEWORK}

  {\HLight
    Algorithms and Data structures\\
    Advanced Machine Learning\\
    Bayesian Methods for Machine Learning\\
    Computational complexity\\
    Information theory\\
    Large-scale machine learning\\
    Natural language processing\\
    Numerical methods\\
    Operating systems\\
  }
\end{cvsection}

\end{leftcolumn}

\begin{rightcolumn}
\begin{cvsection}{EXPERIENCE}
  \cvinfobox{Google Cloud, Virtual Network Dataplane Telemetry}{ 08.2022-present | Software Engineer}
  {Working about packet sampling in virtual network data plane. Main part is instrumentation of virtual
  networking components with packet sampling. \\
  Main technologies are internal DPDK-like framework, C++ in network data plane plus grpc and C++ for packet samples processing. \\
  Main achievements are instrumnetaion of a new virtual networking data plane component,
  2-10\% fast path packet processing speedup by moving expencive syscalls from fast path, introduce framework for 
  simplification of packet drops debugging.}
  
  \cvinfobox{Yandex Search}{ 06.2020-06.2022 | Teamlead}
  {My team was working on a company-wide microservices framework, which serves
  1000 microservices, 10 millions requests per second.\\
  We used C++ for data plane, Python for control plane, and tests/infrastructure.\\
  Main achievements are 99.999\% availability, HTTP transport replacement
  with faster implementation, speedup of distributed tracing over our
  framework from minute to 1-3sec per request.}

  \cvinfobox{Yandex Search}{ 11.2018-06.2020 | Software Engineer}
  {Worked on Search Infrastructure team. C++, Python, grpc, networking.
  Designed and implemented SDCH support for Search Engine Result Page.
  Main part was C++ backend with custom SDCH implementation.\\
  Improved search query preprocessing pipeline. Reimplemented part of business logic with C++, fixed some architectural problems, got speedup.}
  
  \cvinfobox{Google}{07.2018-10.2018 | Software Engineering Intern}
  {Worked on Shopping Data Quality team. Primary C++, MapReduce, Stubby.
   Designed and implemented an infrastructure for intellectual scheduling of merchants review. This infrastructure helped to improve existing scheduling logic
   and allowed the team to migrate from several legacy services.}
  
   \cvinfobox{Yandex Ads}{03.2017-06.2018 | Junior Backend Developer}
  {Designed and implemented distributed log collection and event
   plotting system using internal technologies, including YT — Bigtable-like
   storage with MapReduce framework on top of it. \\
   Used coroutine-based web server and batch requests in order
   to handle required RPS on a small number of servers and not to overload
   other services. \\
   This feature enabled the frontend team to log any client-side
   event (usually JS code in a browser) and plot number of any events on
   clients in real time.}

  \cvinfobox{Mail.ru Group, Tarantool}
  {02.2016-12.2016 | Software Engineering Intern}
  {Tarantool is an in-memory NoSQL database implemented in C with some C++.\\
   Implemented a proof of concept version of SQL in Tarantool.
   We combined SQLite parser and execution engine with Tarantool internal structures (B+*-tree).\\
   Fixed some bugs in Tarantool test engine.\\
   Added intrusive heap to Tarantool data structures library.}
\end{cvsection}

\begin{cvsection}{COMPETITIVE PROGRAMMING}
  \cvinfobox{Russian National Olympiad in Informatics}
  {2012 | prizewinner}
  {}
  \cvinfobox{Russian National Olympiad in Informatics}
  {2011 | prizewinner}
  {}
\end{cvsection}

\end{rightcolumn}
\end{paracol}

\end{document}
